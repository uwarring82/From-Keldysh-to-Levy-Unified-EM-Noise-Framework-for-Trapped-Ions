\section{Spatial and Temporal Coherence Effects}
\label{sec:spatial-coherence}
\begin{tcolorbox}[title=Fast scatterers: spatial coherence effects]
For a moving scatterer with trajectory $\mathbf{r}_k(t)=\mathbf{r}_0+\mathbf{v}t$, the current density is
\[
\mathbf{j}_k(\mathbf{r},t) = \mathbf{j}^{(0)}_k(t)\,\delta^{(3)}(\mathbf{r}-\mathbf{r}_k(t)).
\]
The force spectrum becomes
\[
F_k(\omega) = q \int dt\, e^{\mathrm{i}\omega t}\,\mathbf{G}(\mathbf{r}_0,\mathbf{r}_k(t);\omega)\cdot\mathbf{j}^{(0)}_k(t).
\]

\textbf{Velocity regimes:}
\begin{itemize}\itemsep0.2em
  \item Thermal background ($v \sim 500$~m/s, $\tau_{\rm coll}\sim 10$~ps): $v\tau_{\rm coll}\sim 5$~nm $\ll L_{\rm trap}$ $\to$ point-impulse limit.
  \item Fast cosmics ($v\sim 0.1c$, $\tau_{\rm coll}\sim 1$~ps): $v\tau_{\rm coll}\sim 30~\mu$m $\sim L_{\rm trap}$ $\to$ extended-coherence regime.
\end{itemize}

\textbf{Impact:} Spatial variation of $\mathbf{G}$ introduces directional coupling, $|F_k(\omega)|^2 \propto |\hat{\mathbf{v}}\cdot\nabla\mathbf{G}|^2$, breaking isotropy.

\textbf{Observable:} Anisotropic heating as a function of trap orientation relative to a collimated beam or fast flux (see Fig.~\ref{fig:trajectory_coherence}).
\end{tcolorbox}

\begin{figure}[t]
  \centering
  \begin{tikzpicture}[x=1cm,y=1cm,>=Stealth]
  \small
  \def\W{4.6}
  \def\H{6.2}
  \def\gap{3.2}
  \node at (0,0) {};
  \draw[fill=gray!20] (-1.5,  \gap/2+0.6) rectangle (\W-1.5,  \gap/2+1.2);
  \draw[fill=gray!20] (-1.5, -\gap/2-1.2) rectangle (\W-1.5, -\gap/2-0.6);
  \node[font=\footnotesize] at (\W-1.5+0.8,  \gap/2+0.9) {Electrodes};
  \node[font=\footnotesize] at (\W-1.5+0.8, -\gap/2-0.9) {Electrodes};
  \begin{scope}
    \clip (-1.5, -\H/2) rectangle (\W-1.5, \H/2);
    \foreach \a in {0.8,1.2,1.6,2.0,2.4}{
      \draw[gray!55, decorate, decoration={random steps, segment length=2mm, amplitude=0.6mm}]
        plot[smooth cycle, tension=1] coordinates{
          (-0.8,0) (0.2,\a) (1.2,0.4) (2.0,-0.2) (1.0,-\a) (-0.1,-0.4)};
      \draw[gray!40, decorate, decoration={random steps, segment length=2mm, amplitude=0.5mm}]
        plot[smooth cycle, tension=1] coordinates{
          (0.0,0.2) (0.9,\a+0.4) (2.2,0.6) (3.0,-0.4) (2.0,-\a-0.4) (0.7,-0.6)};
    }
  \end{scope}
  \node[font=\footnotesize, anchor=south] at (-1.2, \H/2-0.2) {$\mathbf{G}(\mathbf{r})$ (spatial structure)};
  \filldraw[fill=white, draw=black, line width=0.6pt] (0.7,0) circle (0.08);
  \node[font=\footnotesize, anchor=west] at (0.85,0) {$\mathbf{r}_0$};
  \draw[thick, -Stealth] (-0.8, -0.4) -- ++(0.55,0.12);
  \node[font=\footnotesize, align=left, anchor=east] at (-1.2,0.5)
    {Thermal: $v\tau_{\rm coll} \ll L_{\rm trap}$\\Point-like impulse \checkmark};
  \draw[thick, dashed, -Stealth] (2.3, -1.3) -- (3.6, 1.1);
  \foreach \t in {0.15,0.38,0.62,0.85}{
    \fill[black!70] ($(2.3, -1.3)!{\t}!(3.6,1.1)$) circle (0.06);
  }
  \node[font=\footnotesize, align=left, anchor=west] at (3.7, 0.5)
    {Cosmic: $v\tau_{\rm coll} \sim L_{\rm trap}$\\Extended coherence, directional $|\hat{\mathbf{v}}\!\cdot\!\nabla\mathbf{G}|^2$};
  \draw[very thin] (-1.2,-\gap/2) -- (-1.2,\gap/2);
  \draw[-{Stealth[length=2mm]}] (-1.2,\gap/2) -- ++(-0.3,0);
  \draw[-{Stealth[length=2mm]}] (-1.2,-\gap/2) -- ++(-0.3,0);
  \node[font=\footnotesize, align=center, anchor=east] at (-1.6,0) {$L_{\rm trap}\ \sim\ \mathrm{mm}$};
  \draw[very thin] (\W-1.0,-1.2) -- (\W-1.0,1.2);
  \draw[-{Stealth[length=2mm]}] (\W-1.0,1.2) -- ++(0.3,0);
  \draw[-{Stealth[length=2mm]}] (\W-1.0,-1.2) -- ++(0.3,0);
  \node[font=\footnotesize, align=center, anchor=west] at (\W-0.6,0)
    {$v\tau_{\rm coll}$: nm (thermal) vs.\ $\mu$m (fast)};
\end{tikzpicture}

  \caption{Representative geometry for extended trajectories producing anisotropic heating signatures.
  The path length $v\tau_{\rm coll}$ relative to trap dimensions $L_{\rm trap}$ determines whether the point-impulse approximation holds.}
  \label{fig:trajectory_coherence}
\end{figure}
