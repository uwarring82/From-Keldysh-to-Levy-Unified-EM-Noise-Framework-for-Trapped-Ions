\section{Inference Protocol}
\label{sec:inference-protocol}
\paragraph{Recommended stroboscopic sequence.}
\begin{enumerate}\itemsep0.2em
  \item Prepare motional ground state $\lvert n=0\rangle$ (or a calibrated thermal $\langle n\rangle$).
  \item Wait time $\tau$ (free evolution under noise).
  \item Read out $n$ via sideband thermometry or time-of-flight.
  \item Repeat $N$ times $\Rightarrow$ histogram $P(n\,|\,\tau)$ and time series $n(t)$.
\end{enumerate}

\begin{table}[h]
  \centering
  \renewcommand{\arraystretch}{1.15}
  \begin{tabularx}{0.9\textwidth}{>{\raggedright\arraybackslash}X>{\raggedleft\arraybackslash}p{4.2cm}}
    \hline
    \textbf{Target quantity} & \textbf{Required shots/events (order-of-magnitude)} \\
    \hline
    Detect $\kappa_3 \neq 0$ at $3\sigma$ & $\sim 10^4$ stroboscopic shots \\
    Reconstruct $\nu(\Delta n)$ (5 bins)  & $\sim 10^5$ shots \\
    Test heavy-tail exponent $\alpha$      & $\sim 10^6$ events \\
    Allan variance rollover near $1/\lambda$ & hours of continuous trace \\
    Sideband asymmetry vs.\ $\omega$ map   & $\sim 10^2$ frequency scans \\
    \hline
  \end{tabularx}
\end{table}

\subsection{Stroboscopic measurement: detailed protocol}
\paragraph{Definition.}
A \emph{stroboscopic measurement} consists of repeated cycles
\[
  \text{Prepare} \to \text{Wait } \tau \to \text{Readout} \to \text{Reset}
\]
with fixed waiting time $\tau$ and $N$ independent repetitions.
The observable is the phonon number $n$ after free evolution under noise.

\paragraph{Pulse sequence and timing.}
\begin{enumerate}
  \item \textbf{Doppler cooling} ($t_{\rm cool} \sim 1{-}2\,\text{ms}$): Bring the ion near the ground state via red-detuned cooling.
  \item \textbf{Resolved-sideband cooling} ($t_{\rm RSBC} \sim 1$~ms): Achieve $n < 0.1$ by sideband pulses and optical pumping.
  \item \textbf{Idle/noise evolution} ($t=\tau$): Allow environmental noise to act.
  \item \textbf{Readout}: Use blue-sideband ratio or time-of-flight to infer $n$.
  \item \textbf{Reset}: Re-cool before the next cycle.
\end{enumerate}

\paragraph{Bayesian inference.}
Given histograms $P(n\,|\,\tau)$ for multiple waiting times, we infer model parameters $\Theta = \{\lambda, \alpha, S_F(\omega)\}$ via hierarchical Bayesian fits.
Posterior sampling (e.g., Hamiltonian Monte Carlo) yields credible intervals and allows evidence comparisons between models.
