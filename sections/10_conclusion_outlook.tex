%==================================================
%  Section 10: Conclusions and Outlook
%==================================================
\section{Conclusions and Outlook}
\label{sec:conclusion}

We have presented a first-principles theoretical framework that unifies
two traditionally distinct regimes of ion heating---continuous Gaussian diffusion
and discrete Poissonian jumps---within a single Lévy–Khintchine master equation
derived from the Keldysh formalism.
This formulation connects microscopic current fluctuations in the environment
to experimentally measurable motional heating rates through the trap's
electromagnetic Green tensor.

The resulting framework provides a transparent hierarchy:
source statistics determine the form of the noise generator,
while the electromagnetic Green tensor mediates how those sources couple to the ion.
This approach clarifies why different physical processes---
technical Johnson noise, surface charge fluctuations, or residual-gas collisions---
produce distinct spectral and statistical signatures.

The model's applicability is limited by the assumptions summarized in
Sec.~\ref{sec:model_assumptions}.
Within that domain, it provides a rigorous and extensible foundation
for analyzing noise in trapped-ion systems used for quantum information
and precision metrology.

Future directions include:
\begin{itemize}
  \item Extending the formalism to correlated multi-ion motion,
        where the electromagnetic Green tensor becomes a matrix coupling multiple spatial modes.
  \item Developing non-Markovian generalizations to describe noise sources
        with long temporal memory such as \(1/f\) patch fluctuations.
  \item Incorporating heavy-tailed Lévy processes to model rare but large
        energy-transfer events and exploring their experimental observability.
  \item Implementing numerical validation of the inference protocol
        proposed in Sec.~\ref{sec:inference_protocol} using synthetic data.
\end{itemize}

Together, these steps will strengthen the framework's role as a quantitative bridge
between microscopic theory and experimental diagnostics of noise
in complex trapping architectures.
