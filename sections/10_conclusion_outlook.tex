%==================================================
%  Section 10: Conclusions and Outlook
%==================================================
\section{Conclusions and Outlook}
\label{sec:conclusion}

We have unified continuous Gaussian diffusion and discrete Poissonian jumps within a single Lévy–Khintchine master equation that connects microscopic current fluctuations to experimentally measurable motional decoherence through the trap's electromagnetic Green tensor. This provides a transparent hierarchy: source statistics determine the noise generator form, while the Green tensor mediates coupling. The framework clarifies why different physical processes produce distinct spectral and statistical signatures, and provides concrete experimental protocols (Table~1, Sec.~8) for discriminating mechanisms in mixed scenarios.
The categorization is exhaustive within the approximations of Sec.~6 (harmonic confinement, Markovian bath, stationary statistics), providing a rigorous foundation for noise analysis in trapped-ion quantum information and precision metrology.

\paragraph{Future directions.} Natural extensions include: (i) multi-ion correlated motion, where $\mathbf{G}$ becomes a matrix coupling spatial modes; (ii) non-Markovian generalizations for 1/f patch fluctuations with long temporal memory; (iii) quantitative heavy-tailed Lévy models for rare large-kick events; and (iv) numerical validation using synthetic data. Together, these will strengthen the framework's role as a quantitative bridge between microscopic theory and experimental diagnostics.