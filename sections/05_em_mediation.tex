\section{Electromagnetic Mediation Pathways}
Electromagnetic interactions mediate every noise source to the trapped ion.
Starting from stochastic current sources $\mathbf{J}(\mathbf{r},t)$, the electric field at the ion position is filtered by the dyadic Green tensor $\mathbf{G}(\mathbf{r}_0,\mathbf{r};\omega)$:
\begin{equation}
\mathbf{E}(\mathbf{r}_0,\omega) = \mathrm{i}\mu_0\omega \int d^3 r\, \mathbf{G}(\mathbf{r}_0,\mathbf{r};\omega) \mathbf{J}(\mathbf{r},\omega).
\end{equation}
Regardless of the microscopic origin of $\mathbf{J}$, the force spectrum $S_F(\omega)$ entering the master equation is
\begin{equation}
S_F(\omega) = q^2 \int d^3 r d^3 r'\, G_{x\alpha}(\mathbf{r}_0,\mathbf{r};\omega) S_{JJ}^{\alpha\beta}(\mathbf{r},\mathbf{r}';\omega) G^{*}_{x\beta}(\mathbf{r}_0,\mathbf{r}';\omega).
\end{equation}

\paragraph{Practical modeling.}
Trap geometries are encoded via $\mathbf{G}$, typically computed with boundary-element or finite-element solvers.
Once tabulated, $\mathbf{G}$ enables rapid evaluation of candidate noise sources:
\begin{itemize}
  \item \textbf{Technical fields:} Johnson noise or control electrode pickup yields Gaussian $S_{JJ}$.
  \item \textbf{Gas collisions:} Charged or neutral scatterers contribute impulses parameterized by cross-sections (Sec.~\ref{sec:scattering-models}).
  \item \textbf{Surface dipoles:} Patch potentials manifest as correlated dipole moments with colored spectra.
\end{itemize}

\paragraph{Connection to figures.}
Figure~\ref{fig:em_mediation} visualizes the mediation network, while spatial coherence effects for extended trajectories are shown in Fig.~\ref{fig:trajectory_coherence}.
