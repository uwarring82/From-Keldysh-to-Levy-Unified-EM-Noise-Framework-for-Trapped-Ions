\section{Theoretical Framework: From Keldysh to Lévy}
\label{sec:theory_foundations}

\subsection{System--Bath Setup}
We model a single trapped ion of charge $q$ and mass $m$, confined harmonically at the secular frequency $\omega_t$ along $x$.
The ion couples to the fluctuating electric field $\mathbf{E}(\mathbf{r}_0,t)$ at its equilibrium position $\mathbf{r}_0$, generated by stochastic currents $\mathbf{J}(\mathbf{r},t)$ in the environment.
The Hamiltonian is
\begin{equation}
H = \frac{p^2}{2m} + \tfrac{1}{2} m \omega_t^2 x^2 - q x E_x(\mathbf{r}_0,t).
\end{equation}
The field obeys Maxwell's equations with sources $\mathbf{J}$, and the ion couples linearly via $H_{\text{int}} = -q x E_x$.

\subsection{Electromagnetic Green Tensor}
The electric field is expressed through the electromagnetic Green tensor $G(\mathbf{r}_0,\mathbf{r};\omega)$, written in dyadic form as $\mathbf{G}$, relating currents to fields in frequency space,
\begin{equation}
\mathbf{E}(\mathbf{r}_0,\omega) = \mathrm{i}\mu_0 \omega \int d^3r\, \mathbf{G}(\mathbf{r}_0,\mathbf{r};\omega) \mathbf{J}(\mathbf{r},\omega).
\end{equation}
Here $\mathbf{G}$, the dyadic representation of $G$, satisfies the Helmholtz equation
\begin{equation}
\left[\nabla\times\nabla\times - \frac{\omega^2}{c^2}\varepsilon(\mathbf{r},\omega)\right]\mathbf{G}(\mathbf{r},\mathbf{r}';\omega) = \delta(\mathbf{r}-\mathbf{r}')\mathbf{I}.
\end{equation}
The permittivity $\varepsilon(\mathbf{r},\omega)$ encodes electrode/dielectric response; in practice $G$ is computed numerically for trap geometry.

\subsection{Electromagnetic Mediation Pathways}
Electromagnetic interactions mediate every noise source to the trapped ion.
Starting from stochastic current sources $\mathbf{J}(\mathbf{r},t)$, the electric field at the ion position is filtered by the dyadic Green tensor $\mathbf{G}(\mathbf{r}_0,\mathbf{r};\omega)$:
\begin{equation}
\mathbf{E}(\mathbf{r}_0,\omega) = \mathrm{i}\mu_0\omega \int d^3 r\, \mathbf{G}(\mathbf{r}_0,\mathbf{r};\omega) \mathbf{J}(\mathbf{r},\omega).
\end{equation}
Regardless of the microscopic origin of $\mathbf{J}$, the force spectrum $S_F(\omega)$ entering the master equation is
\begin{equation}
S_F(\omega) = q^2 \int d^3 r d^3 r'\, G_{x\alpha}(\mathbf{r}_0,\mathbf{r};\omega) S_{JJ}^{\alpha\beta}(\mathbf{r},\mathbf{r}';\omega) G^{*}_{x\beta}(\mathbf{r}_0,\mathbf{r}';\omega).
\end{equation}

\paragraph{Practical modeling.}
Trap geometries are encoded via $\mathbf{G}$, typically computed with boundary-element or finite-element solvers.
Once tabulated, $\mathbf{G}$ enables rapid evaluation of candidate noise sources:
\begin{itemize}
  \item \textbf{Technical fields:} Johnson noise or control electrode pickup yields Gaussian $S_{JJ}$.
  \item \textbf{Gas collisions:} Charged or neutral scatterers contribute impulses parameterized by cross-sections (Sec.~\ref{sec:scattering-models}).
  \item \textbf{Surface dipoles:} Patch potentials manifest as correlated dipole moments with colored spectra.
\end{itemize}

\paragraph{Connection to figures.}
Figure~\ref{fig:em_mediation} visualizes the mediation network, while spatial coherence effects for extended trajectories are shown in Fig.~\ref{fig:trajectory_coherence}.

\subsection{Keldysh Action and Force-Force Correlator}
The ion+EM+bath system follows a Keldysh action doubling fields on forward/backward contours [13]. Integrating out electromagnetic modes yields an effective action for the ion with force-force correlator
\begin{equation}
S_F(\omega) = q^2 \int d^3r d^3r' G_{x\alpha}(\mathbf{r}_0, \mathbf{r}; \omega) S^{\alpha\beta}_{JJ}(\mathbf{r}, \mathbf{r}'; \omega) G^*_{x\beta}(\mathbf{r}_0, \mathbf{r}'; \omega).
\end{equation}
\textbf{Key result}: All environmental noise enters via $S_F(\omega)$, the current spectrum $S_{JJ}$ filtered by trap response $|G(\omega)|^2$. Distinct source statistics (Gaussian vs Poisson) produce distinct dynamics, as detailed below.

\subsection{Gaussian Bath $\to$ Diffusion}
For dense, weakly coupled baths, $\mathbf{J}$ is Gaussian with correlations $S_{JJ}$.
The effective master equation is of Lindblad form
\begin{equation}
\dot{\rho} = -\mathrm{i}[H_0,\rho] + \Gamma_\uparrow \mathcal{D}[a^\dagger]\rho + \Gamma_\downarrow \mathcal{D}[a]\rho,
\end{equation}
with rates $\Gamma_{\uparrow,\downarrow} \propto S_F(\pm\omega_t)$.
The heating rate is
\begin{equation}
\Gamma_{\text{heat}} = \Gamma_\uparrow - \Gamma_\downarrow = \frac{x_0^2}{\hbar^2}\left[S_F(+\omega_t) - S_F(-\omega_t)\right],
\end{equation}
with $x_0=\sqrt{\hbar/(2m\omega_t)}$.
In equilibrium, detailed balance gives $S_F(-\omega_t)/S_F(+\omega_t)=e^{-\hbar\omega_t/k_BT}$.

\subsection{Poisson Bath $\to$ Jumps}
For sparse scatterers, $\mathbf{J}$ is a sum of impulses $\mathbf{j}_k(t-t_k)$.
The effective dynamics are compound-Poisson: each event imparts momentum $\Delta p$, with rate $\lambda$ and distribution $\nu(\Delta p)$.
The generator is
\begin{equation}
\mathcal{J}_P\rho = \lambda \int d(\Delta p)\, \nu(\Delta p)\,\left[e^{-\mathrm{i}\Delta p x/\hbar}\rho e^{+\mathrm{i}\Delta p x/\hbar}-\rho\right].
\end{equation}

\subsection{Unified Lévy–Khintchine Form}
The full ion master equation is thus
\begin{equation}
\dot{\rho} = -\mathrm{i}[H_0,\rho] + \mathcal{D}_G[\rho] + \mathcal{J}_P[\rho],
\label{eq:levy-box}
\end{equation}
a Lévy–Khintchine generator: Gaussian diffusion $\mathcal{D}_G$ plus compound-Poisson jumps $\mathcal{J}_P$.

\subsection{Universal L\'evy--Khintchine Structure}
\label{subsec:universal-LK}
The master equation in Eq.~\eqref{eq:levy-box} realizes the general L\'evy--Khintchine decomposition for translation-covariant quantum Markov semigroups, as established in the structural theorems of Holevo and Vacchini~\cite{Holevo1993,VacchiniJMathPhys2001} and reviewed in Ref.~\cite{BreuerPetruccione2002}.
In brief, any linear system--environment coupling $H_{\text{int}}=\hat{A}\otimes\hat{B}$ splits uniquely into a Gaussian (diffusive) contribution governed by second moments of the bath operators and a jump (Poissonian) contribution determined by the associated L\'evy measure.
This decomposition does not assume a particular system Hilbert space: the same generator governs discrete variables (e.g.~qubit dephasing with $\hat{A}=\sigma_z$) and continuous variables (e.g.~motional coupling with $\hat{A}=\hat{x}$), as long as the dynamics remain translation covariant.
It also accommodates mixtures of channels, where correlated baths produce simultaneous diffusion and jumps, and provides the canonical starting point for process-agnostic noise classification.
\emph{Scope of this work:} we specialize to trapped-ion motion ($\hat{A}=\hat{x}$) under electromagnetic noise and develop explicit connections to experimental observables (regime classifier $\lambda\tau$, discriminants, and inference protocols).
A comprehensive treatment of discrete-variable platforms is beyond our scope; readers seeking qubit-focused perspectives are referred to Refs.~\cite{Paladino2014,Cywinski2008}.

\paragraph{Crossover regimes and dominance criteria.}
The relative importance of diffusion versus jumps is set by dimensionless ratios:
\begin{equation}
\frac{\text{jump heating}}{\text{Gaussian heating}}
\sim \frac{\lambda \langle (\Delta p)^2 \rangle}{x_0^2 S_F(\omega_t)/\hbar^2},
\quad
\frac{\text{events per measurement}}{\text{one}}
\sim \lambda \tau.
\end{equation}
Three regimes emerge naturally:
\begin{itemize}
\item \textbf{Diffusive limit} ($\lambda \tau \gg 1$, small $\langle(\Delta p)^2\rangle$):
  Jumps coalesce into effective Gaussian noise;
  $\kappa_3, \kappa_4 \to 0$;
  Allan variance $\sigma_A^2(\tau) \propto \tau^{-1}$.
\item \textbf{Intermediate regime} ($\lambda \tau \sim 1{-}10$):
  Both terms in Eq.~\eqref{eq:levy-box} contribute comparably;
  non-Gaussian cumulants measurable ($\kappa_3/\kappa_2^{3/2} \sim 0.3{-}1$);
  Allan variance shows rollover near $\tau \sim 1/\lambda$
  (Table~\ref{tab:discriminants}, Sec.~\ref{sec:discriminants_summary}).
\item \textbf{Poisson limit} ($\lambda \tau \ll 1$, large $\langle(\Delta p)^2\rangle$):
  Discrete, resolvable jumps dominate;
  waiting-time statistics exponential;
  $P(\Delta n)$ determined by $\nu(\Delta n)$ directly.
\end{itemize}
Experimentally, the intermediate regime is often the most informative:
it separates mechanisms via higher-order statistics while still accumulating
measurable events on laboratory timescales.
Table~\ref{tab:discriminants} provides experimentally accessible discriminants optimized for this intermediate regime, where neither limit dominates and model selection becomes nontrivial.

\subsection{From Momentum to Phonon Jumps}
Experimentally, jumps are resolved in motional phonon number $n$.
Momentum transfer $\Delta p$ maps to phonon transitions via
\begin{equation}
\nu(\Delta n) = \int d(\Delta p)\,\nu(\Delta p)\,\sum_{n'} P_{n\to n'}(\Delta p)\, \delta(\Delta n-(n'-n)),
\label{eq:nu_to_deltan}
\end{equation}
with $P_{n\to n'}(\Delta p)=|\langle n'|e^{-\mathrm{i}\Delta p x/\hbar}|n\rangle|^2$.
This connects scattering cross-sections $d\sigma/d\Omega$ to measurable $\nu(\Delta n)$ (see Sec.~\ref{sec:scattering-models}).

\paragraph{Takeaway.}
The ion's heating is universally described by Eq.~\eqref{eq:levy-box}, independent of microscopic mechanism.
The experimentally accessible regime is often \emph{intermediate}, where $\lambda \tau \sim 1{-}10$ and both Gaussian and jump terms contribute measurably.
Section~\ref{sec:observable_signatures} develops observables—cumulants, waiting times, and Allan variance—that discriminate $\mathcal{D}_G$ versus $\mathcal{J}_P$ even when both are present simultaneously.

\subsection{Heavy-Tailed Lévy Processes}

Heavy-tailed Lévy flights extend the compound-Poisson picture to jump-size distributions with divergent variance, $\nu(\Delta p) \propto |\Delta p|^{-1-\alpha}$ for $0 < \alpha < 2$. Because the second moment diverges, the central-limit theorem no longer applies and the motion follows fractional diffusion with mean-square displacement $\langle(\Delta x)^2\rangle \propto t^{2/\alpha}$. 

Such processes require collision mechanisms that enhance \emph{large} momentum transfer. Candidates include:
\begin{itemize}[nosep,leftmargin=*]
\item \textbf{Coulomb scattering on charged dust grains:} Long-range $1/r^2$ potential produces power-law differential cross-section at large angles, unlike the forward-peaked Langevin scattering from induced dipoles.
\item \textbf{Micro-discharge events:} Sudden electrode charging creates impulsive large-momentum kicks with broad distribution.
\item \textbf{Cosmic ray impacts:} High-energy particles transfer momentum far exceeding thermal collisions.
\end{itemize}

Via Eq.~\eqref{eq:nu_to_deltan}, the power-law momentum distribution maps to phonon jumps: $P(\Delta n) \propto |\Delta n|^{-1-\alpha}$. The Allan variance acquires an anomalous slope $\sigma^2_A(\tau) \propto \tau^{-(2/\alpha-1)}$ with non-integer exponent, producing the long tails listed in Table~\ref{tab:discriminants}. The generator replaces the finite-variance compound-Poisson term in Eq.~\eqref{eq:jump_generator} by a fractional Laplacian $-D_\alpha(-\partial_x^2)^{\alpha/2}$.

\paragraph{Experimental relevance.} Heavy-tailed Lévy processes are distinct from the \emph{near-Gaussian} behavior produced by Langevin ion-neutral collisions (Sec.~\ref{sec:langevin}), which enhance small momentum transfer and obey CLT. Heavy-tailed signatures become observable when rare large-kick mechanisms dominate: ultra-low neutral pressures where charged dust encounters outweigh thermal collisions, environments with frequent charging events, or high-radiation facilities. Experimentally distinguishing $\alpha<$2 power-law tails from Langevin fat tails requires measuring $P(\Delta n)$ over 3–4 decades (approx. $10^6$ events) to resolve the asymptotic behavior. A full quantitative treatment is left for future work.