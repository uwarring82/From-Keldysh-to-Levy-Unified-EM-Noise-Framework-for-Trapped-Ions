%==================================================
%  Section 2: Noise Regimes and Motional Responses
%==================================================
\section{Noise Regimes and Motional Responses}
\label{sec:noise_regimes}

Environmental fluctuations couple to a trapped ion’s harmonic motion either
as effectively continuous fields or as discrete impulses.  Both arise from
stochastic current sources $J(\mathbf r,t)$ and are mediated by the trap’s
electromagnetic Green tensor $G(\mathbf r_0,\mathbf r;\omega)$.
In the following, we use the term \emph{motional decoherence} to refer
collectively to all processes by which environmental noise degrades the
purity of the ion’s motional state. This encompasses both \emph{heating}—
energy exchange via resonant force fluctuations—and \emph{dephasing}—
phase diffusion caused by slow potential variations. Where the distinction
matters for understanding spectral origins or experimental signatures,
we refer to heating and dephasing explicitly.

To classify temporal statistics we use the dimensionless product
$\lambda\tau$, which counts independent events within one experimental
interrogation time~$\tau$.  For impulsive processes $\lambda$ denotes the
Poisson rate of discrete events; for continuous Gaussian fields,
$\lambda^{-1}$ represents an effective correlation time $\tau_c$, so that
$\lambda\tau\!\sim\!\tau/\tau_c$ measures the number of independent
fluctuation intervals.  \emph{Crucially, $\lambda\tau$ characterizes event
statistics, not the frequency content of the noise spectrum.}
Heating and dephasing depend on \emph{spectral weight} at different
frequencies, independent of whether the statistics are diffusive or sparse.

%----------------------------------------
\paragraph{Diffusive regime ($\lambda\tau\!\gg\!1$).}
Many independent fluctuations accumulate and the central-limit theorem
applies: force statistics are well described by a Gaussian process.
The ion’s energy changes at a rate set by the near-resonant spectral weight
of the force noise filtered by the mechanical susceptibility
$\chi(\omega)$ associated with the secular frequency $\omega_t$
(derived in Sec.~\ref{sec:theory_foundations}):
\[
\frac{d\langle n\rangle}{dt}
 = \frac{1}{4 m \hbar \omega_t}\!\int_{-\infty}^{\infty}\!
   |\chi(\omega)|^{2} S_F(\omega)\,d\omega,
\qquad S_F(\omega)=q^{2}S_E(\omega).
\]
The process is statistically Gaussian, but this says nothing about phase
coherence: dephasing can be strong if low-frequency spectral weight is large
(e.g.\ $1/f$ noise), and weak if such weight is small.

%----------------------------------------
\paragraph{Intermediate regime ($\lambda\tau\!\sim\!1$).}
Only a few fluctuations occur within~$\tau$.  The motion alternates between
free evolution and discrete perturbations, producing non-Gaussian
statistics with measurable higher-order cumulants and characteristic
roll-overs in the Allan variance.  Each impulse contributes \emph{both}
energy change and a phase shift, so motional decoherence manifests through
coupled energy and phase fluctuations.
This regime is particularly diagnostic of underlying source statistics
(cf.\ Sec.~\ref{sec:discriminants_summary}).

%----------------------------------------
\paragraph{Sparse regime ($\lambda\tau\!\ll\!1$).}
Isolated impulses are resolvable in time with exponential waiting-time
distribution $p(t)=\lambda e^{-\lambda t}$.  Between impulses the oscillator
evolves coherently; each rare impulse applies a finite displacement in
phase space, producing a discrete energy jump together with a phase reset.
The observables are step-like heating and discrete, resolvable phase
discontinuities.

%----------------------------------------
\subsection*{Heating and Dephasing Channels: Spectral View and Operators}

Environmental noise modifies (i) the \emph{energy} of the oscillator
(heating) and (ii) the \emph{phase} of its coherent motion (dephasing).
These effects are controlled by different parts of the spectrum of the
electric field at the ion.

\begin{itemize}
  \item \textbf{Heating (energy exchange).}
  With $H_\mathrm{int}=-q x E(t)$, the heating rate is determined by the
  force-noise spectrum $S_F(\omega)=q^2 S_E(\omega)$ filtered by the
  mechanical susceptibility $|\chi(\omega)|^{2}$.  In the high-$Q$ limit the
  integral above reduces to the familiar near-resonant expression
  $\dot{\langle n\rangle}\!\approx\! (q^{2}/4m\hbar\omega_t)\,S_E(\omega_t)$.
  Off-resonant spectral weight contributes according to
  $|\chi(\omega)|^{2}$ and should not be neglected when linewidths are broad
  or spectra structured.

  \item \textbf{Dephasing (phase diffusion without energy exchange).}
  Slow fluctuations of the trapping potential or stray fields modulate the
  instantaneous frequency
  $\omega_t\!\rightarrow\!\omega_t+\delta\omega(t)$,
  producing a random phase
  $\phi(t)=\int_{0}^{t}\delta\omega(t')\,dt'.$
  The coherence of a motional superposition decays as
  $\exp[-\tfrac{1}{2}\langle(\Delta\phi)^{2}\rangle]$, with
  \[
    \langle(\Delta\phi)^{2}\rangle
      = 2\!\int_{0}^{\infty}\! S_{\delta\omega}(\Omega)
        \!\left(\frac{\sin(\Omega t/2)}{\Omega/2}\right)^{2}\! d\Omega .
  \]
  In realistic traps, however, the potential includes higher-order terms
  beyond the harmonic approximation,
  \[
    U(x) = \tfrac{1}{2}m\omega_t^2x^2 + \beta x^3 + \gamma x^4 + \dots .
  \]
  While the master equation derived in Sec.~\ref{sec:theory_foundations}
  assumes harmonic confinement and linear coupling, anharmonic corrections
  to dephasing rates can be incorporated perturbatively when
  $|\beta|x_0 \ll m\omega_t^2$ and $|\gamma|x_0^2 \ll m\omega_t^2$.
  In this weakly anharmonic regime, small shifts of the equilibrium position
  caused by $\delta E(t)$ change the local curvature experienced by the ion
  and thus the instantaneous secular frequency.
  For cubic terms the frequency shift scales linearly with displacement,
  $\delta\omega(t) \propto \beta\,\delta x(t)$,
  whereas quartic terms contribute quadratically,
  $\delta\omega(t) \propto \gamma\,(\delta x(t))^{2}$.
  These position-dependent curvature changes can substantially enhance
  motional dephasing even when the field noise itself is weak.
  True frequency noise also arises from electric-field gradients or
  parametric variations of the RF or control voltages that modulate the
  confining curvature.  In such cases the spectrum
  $S_{\delta\omega}(\Omega)$ is directly linked to the low-frequency
  component of the field-gradient or control-voltage noise spectrum.
  In the Markov limit $S_{\delta\omega}(0)$ defines an exponential
  dephasing time~$T_{\phi}$, while quasi-static variations produce Gaussian
  temporal decay.
\end{itemize}

\noindent\textbf{Unified impulse picture.}
A single momentum impulse implements the unitary displacement
$U_{\Delta p}=\exp[-i\,\Delta p\,x/\hbar]$,
which simultaneously increases energy and shifts the phase of the coherent
state.  Accordingly, it is incorrect to assign ``kicks~$\rightarrow$~heating''
and ``timing~$\rightarrow$~dephasing'' to separate channels; both arise from
the same operator action, while their \emph{observed balance} is governed by
the spectral content of $S_E(\omega)$ and the trap’s susceptibility.

\medskip
\noindent\textbf{Summary.}
The parameter $\lambda\tau$ organizes temporal statistics
(diffusive~$\leftrightarrow$~sparse), whereas heating and dephasing are set by
\emph{which} parts of the spectrum carry power:
near~$\omega_t$ for heating and near~0 for dephasing.
Either contribution can dominate in any regime.

\subsection{Experimental Context: From Single Shots to Statistical Inference}

To make the regime classification concrete, we first establish the basic experimental workflow familiar to any trapped-ion student.

\paragraph{Single experimental shot.}
A standard motional-state measurement consists of:
\begin{enumerate}[nosep]
\item \textbf{Initialize}: Doppler cool, then resolved-sideband cool to near ground state ($n \lesssim 0.1$)
\item \textbf{Wait}: Idle for duration $\tau$ while environmental noise acts on the ion
\item \textbf{Readout}: Measure phonon number $n$ via sideband thermometry or fluorescence.
\end{enumerate}
The outcome is a single integer $n$ representing the motional quantum number after noise exposure. Typical wait times range from $\tau \sim 10\,\mu$s (studying fast technical noise) to $\tau \sim 100\,$ms (studying slow drifts or rare collision events). The readout itself takes $\sim 1$~ms but does not contribute to the noise exposure.

\paragraph{Building statistics through repetition.}
A single shot provides minimal information. To characterize the noise, the sequence is repeated $N$ times (typically $N = 10^2$ to $10^5$), yielding a histogram $\{n_i\}_{i=1}^N$ from which we extract:
\begin{itemize}[nosep]
\item Mean heating: $\langle n \rangle = N^{-1} \sum_i n_i$
\item Variance: $\sigma^2_n = \langle n^2 \rangle - \langle n \rangle^2$
\item Higher cumulants: skewness $\kappa_3$, kurtosis $\kappa_4$
\item Full distribution: $P(n) = (\text{number of shots with outcome } n)/N$
\end{itemize}
For Gaussian noise, $P(n)$ is approximately Poisson (or thermal) and fully characterized by its mean. For non-Gaussian noise, $\kappa_3 \neq 0$ and the distribution develops tails or structure invisible to $\langle n \rangle$ alone.

\paragraph{Time-series analysis (optional).}
If shot outcomes are recorded sequentially as $\{n(t_1), n(t_2), \ldots\}$ with time stamps, one can additionally compute:
\begin{itemize}[nosep]
\item Waiting times between jumps: $\Delta t_i = t_{i+1} - t_i$ when $|n_{i+1} - n_i| > \text{threshold}$
\item Allan variance: $\sigma^2_A(\tau) = \langle [n(t+\tau) - n(t)]^2 \rangle / 2$ as a function of averaging time
\item Autocorrelation: $C(\tau) = \langle n(t) n(t+\tau) \rangle - \langle n \rangle^2$
\end{itemize}
Time-series methods are particularly powerful in the intermediate and sparse regimes where discrete events produce resolvable structure.

\paragraph{Connecting to the regime parameter $\lambda\tau$.}
The dimensionless product $\lambda\tau$ counts the \emph{typical number of noise events occurring during a single shot of duration $\tau$}:
\begin{itemize}
\item $\lambda$ = characteristic rate of environmental disturbances (e.g., gas collision rate, charging event rate, or effective correlation rate $1/\tau_c$ for continuous field noise)
\item $\tau$ = your chosen wait duration in the pulse sequence
\item $\lambda\tau$ = expected number of independent noise events per shot
\end{itemize}

\textbf{Physical interpretation}:
\begin{itemize}
\item \textbf{$\lambda\tau \gg 1$} (diffusive regime): Each shot accumulates many noise events—your histogram $P(n)$ looks smooth, approximately Gaussian or thermal. Shot-to-shot fluctuations follow central-limit statistics. \emph{Example}: Technical field noise with $\tau_c \sim 1\,\mu$s and $\tau = 1\,$ms gives $\lambda\tau \sim 10^3$.

\item \textbf{$\lambda\tau \sim 1$} (intermediate regime): Each shot experiences zero, one, or a few events. Your histogram $P(n)$ develops visible skewness and excess kurtosis. Some shots heat significantly, others barely at all. Time-series show clustering. \emph{Example}: Gas collisions at $P = 10^{-10}\,$mbar with $\tau = 10\,$ms gives $\lambda\tau \sim 1{-}5$ depending on species and trap size.

\item \textbf{$\lambda\tau \ll 1$} (sparse regime): Most shots show $n \approx 0$ (no event), but occasionally a shot exhibits a large jump ($\Delta n \sim 5{-}20$). Waiting times between jumps are exponentially distributed. \emph{Example}: Cosmic ray impacts or rare charging events with $\lambda \sim 0.01\,$Hz and $\tau = 10\,$ms gives $\lambda\tau \sim 10^{-4}$.
\end{itemize}

%\begin{figure}[t]
%  \centering
%  \includegraphics[width=\textwidth]{figures/experimental_workflow}
%  \caption{\textbf{From single shots to regime classification.}
%  \textbf{Top}: Standard pulse sequence—cool, wait duration $\tau$, readout phonon number $n$.
%  \textbf{Middle}: Example histograms $P(n)$ for three regimes after $N=10^4$ repetitions with same mean $\langle n \rangle = 5$ but different statistics.
%  \textbf{Left} ($\lambda\tau \gg 1$): Smooth Gaussian-like distribution from many small kicks.
%  \textbf{Center} ($\lambda\tau \sim 1$): Skewed distribution with visible higher cumulants.
%  \textbf{Right} ($\lambda\tau \ll 1$): Bimodal with peak at $n=0$ (no event) and exponential tail (rare jumps).
%  \textbf{Bottom}: Time traces $n(t)$ showing diffusive vs. jump-like evolution.}
%  \label{fig:experimental_workflow}
%\end{figure}

The regime parameter $\lambda\tau$ provides a simple heuristic: estimate the noise event rate (from pressure gauges, technical noise measurements, or prior studies), multiply by your planned wait time, and the result tells you which observables to focus on (Table~\ref{tab:discriminants}).

\paragraph{Experimental knobs for tuning $\lambda\tau$.}
A key advantage of this framework is that $\lambda\tau$ can be tuned experimentally:
\begin{itemize}[nosep]
\item \textbf{Change $\tau$}: Use shorter wait times to probe the diffusive regime of a collision-dominated system, or longer wait times to accumulate more events for better statistics
\item \textbf{Change $\lambda$}: Vary background pressure (for gas collisions), reduce technical noise (for field fluctuations), or operate at different temperatures
\item \textbf{Scan both}: Systematic variation of $\tau$ at fixed $\lambda$ produces the Allan variance $\sigma^2_A(\tau)$, whose functional form distinguishes regimes
\end{itemize}

With this experimental context established, we now turn to the microscopic theory that predicts which value of $\lambda$ and which functional form of $P(n)$ to expect from different noise sources.

\begin{tcolorbox}[title=Experimental Terminology]
\begin{description}[leftmargin=3em,style=nextline,nosep]
\item[Shot] One complete cycle: cool $\to$ wait $\tau$ $\to$ readout $\to$ reset. Yields one phonon number $n$.
\item[Wait time $\tau$] Duration during which environmental noise acts on the ion (controlled parameter).
\item[Repetitions $N$] Number of independent shots performed to build histogram $P(n)$.
\item[Heating rate] Mean phonon gain per unit time: $\dot{n} = \langle n \rangle / \tau$ (units: quanta/s).
\item[Event rate $\lambda$] Characteristic frequency of environmental disturbances (Hz). For collisions: $\lambda = n_g \sigma v$. For continuous noise: $\lambda \sim 1/\tau_c$.
\item[Regime parameter] $\lambda\tau$ = expected number of noise events per shot (dimensionless).
\item[Stroboscopic measurement] Repeated fixed-$\tau$ shots; contrasts with continuous monitoring.
\item[Allan variance] Two-sample variance $\sigma^2_A(\tau) = \langle [n(t{+}\tau) - n(t)]^2 \rangle/2$ vs. averaging time.
\end{description}
\end{tcolorbox}