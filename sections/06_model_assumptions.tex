%==================================================
%  Section 6: Model Assumptions and Validity Domain
%==================================================
\section{Model Assumptions and Validity Domain}
\label{sec:model_assumptions}

The unified L\'evy--Keldysh framework presented in the preceding sections relies on several
physical and mathematical approximations that define its domain of validity.
We summarize these explicitly below to delineate the scope of applicability
and to prevent over-interpretation beyond their justified range.

\subsection{Linear Coupling Approximation}
The system--bath interaction Hamiltonian is taken to be
\begin{equation}
H_\mathrm{int}(t) = -q\,x(t)\,E_x(r_0,t),
\end{equation}
which assumes a linear coupling between the ion's displacement and the local electric field.
This approximation is accurate when the ion's motion remains small compared to the spatial
scale over which the field varies, \(x \ll r_0\).  
In regimes with strong anharmonic confinement or large motional excitation,
higher-order coupling terms \(x^2 E'(r_0,t)\) may contribute and modify the effective noise spectrum.

\subsection{Markovian Bath Assumption}
The derivation of the Lévy–Khintchine generator presumes a memoryless environment,
where the bath correlation time \(\tau_c\) is short compared to the ion's motional period
set by the secular frequency \(\omega_t\) (period \(\omega_t^{-1}\)).
Under this condition, the noise statistics can be encoded in a time-independent generator.
For slow environmental processes such as \(1/f\) charge-patch fluctuations,
a non-Markovian generalization with an explicit memory kernel \(K(t-t')\)
would be required to accurately capture temporal correlations.

\subsection{Stationarity and Ergodicity}
The framework assumes stationary noise statistics,
i.e.\ correlation functions depend only on time differences.
Ergodicity ensures that ensemble averages coincide with long-time averages,
enabling experimental inference from single-trajectory measurements.

\subsection{Trap Geometry and Green Tensor}
All spatial field correlations are expressed through the electromagnetic Green tensor
\(G(r,r';\omega)\), which is geometry-dependent and, in practice,
evaluated numerically for a given electrode configuration.
Uncertainties in geometry modeling directly propagate into predictions
of heating rates and noise spectra.  
Validation of \(G\) against experimental benchmarks is therefore essential
for quantitative accuracy.

\subsection{Summary of Validity Domain}
The present formalism accurately describes single-ion motional heating
in the linear, Markovian, and stationary limits.
Extensions to multi-ion or correlated environments are conceptually straightforward
but lie beyond the current manuscript's scope and will be pursued in future work.
