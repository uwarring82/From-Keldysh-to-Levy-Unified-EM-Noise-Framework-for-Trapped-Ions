\section{Introduction}
Trapped ions are among the most sensitive probes of weak electromagnetic fields.
Understanding and controlling their motional heating has been central since the earliest quantum logic and precision spectroscopy experiments~\cite{Turchette2000,Wineland1998}.
Traditionally, heating is modeled through spectral densities of electric field fluctuations $S_E(\omega)$, inferred by relating the ion's motional excitation rate to the environmental noise spectrum.
This framework has successfully explained technical noise sources, Johnson noise in electrodes, and surface-potential fluctuations~\cite{Brownnutt2015}.
However, a complementary mechanism arises from discrete collision events: background gas flybys or scattering particles that impart sudden momentum kicks.
These cannot be captured by Gaussian noise models alone.
Instead, they produce intermittent heating signatures that are invisible to $S_E(\omega)$ alone.

We present here a unifying framework where both continuous noise and discrete collisions are subsumed into the language of stochastic currents $\mathbf{J}(\mathbf{r},t)$ mediated to the ion through the electromagnetic Green tensor $\mathbf{G}(\mathbf{r}_0,\mathbf{r};\omega)$.
As illustrated schematically in Fig.~\ref{fig:em_mediation}, all differences reduce to the statistical character of the source currents: dense Gaussian baths vs.~sparse Poisson impulses.

\begin{figure}[t]
  \centering
  \begin{tikzpicture}[x=1cm,y=1cm,>=Stealth, node distance=2.cm]
  \small
  \tikzset{
    ion/.style={circle, minimum size=7mm, inner sep=0pt, draw=black, fill=blue!35},
    block/.style={draw, rounded corners, fill=gray!12, inner sep=6pt},
    annot/.style={align=left, text width=3.4cm},
    mathnote/.style={font=\footnotesize, inner sep=1pt}
  }
  \node[ion,label=above:{Ion}] (ion) {};
  \draw[line width=0.5pt] ([yshift=-0.35cm]ion.south) -- ++(0,-0.15)
    decorate[decoration=zigzag]{ -- ++(0,-0.5) } -- ++(0,-0.15);
  \node[below=1.4cm of ion, font=\footnotesize] {harmonic confinement};
  \node[block, minimum width=3.3cm, minimum height=2.6cm, right=of ion] (G) {};
  \begin{scope}
    \clip (G.south west) rectangle (G.north east);
    \fill[pattern=north west lines, pattern color=gray!60]
      ($(G.south west)+(0.1,0.1)$) rectangle ($(G.north east)+(-0.1,-0.1)$);
  \end{scope}
  \node at (G) {$\mathbf{G}(\mathbf{r}_0,\mathbf{r};\omega)$};
  \node[mathnote, below=0.35cm of G] {EM response / trap geometry};
  \node[right=of G] (src) {};
  \node[annot, above=1.0 cm of src, anchor=west] (denseLab) {Dense sources $\rightarrow$ Gaussian $S_{JJ}$\\[2pt] $\mathcal{D}_G[\rho]$};
  \foreach \i in {0,...,8}{
    \draw[-{Stealth[length=2mm]}, line width=0.3pt]
      ($(denseLab.east)+(0.2,{-0.9+0.2*\i})$) -- ++(0.6,0.05);
    \fill[black!65] ($(denseLab.east)+({0.1+0.15*mod(\i,3)},{-0.92+0.2*\i})$) circle (0.035);
  }
  \node[annot, below=0.8cm of src, anchor=west] (sparseLab) {Sparse impulses $\rightarrow$ Poisson $\nu(\Delta p)$\\[2pt] $\mathcal{J}_P[\rho]$};
  \foreach \p/\dx in {0.5/0.9, -0.2/0.6, 0.1/1.1}{
    \fill[black!65] ($(sparseLab.east)+(\dx,\p)$) circle (0.07);
    \draw[-{Stealth[length=2.3mm]}] ($(sparseLab.east)+(\dx-0.25,\p-0.05)$) -- ++(0.25,0.05);
  }
  \draw[<->, line width=0.7pt] (ion) -- node[above, mathnote] {$\mathbf{F}(t)=q\mathbf{E}$} (G.west);
  \draw[<->, line width=0.7pt] (G.east) -- node[above, mathnote] {mediated by $\mathbf{J}$} ($(src)+(0.9,0)$);
\end{tikzpicture}

  \caption{\textbf{Unified EM-mediated noise framework.}
  All heating mechanisms—technical field noise and collisional impulses—couple to the trapped ion through the electromagnetic Green tensor $\mathbf{G}(\mathbf{r}_0,\mathbf{r};\omega)$ acting on stochastic current sources $\mathbf{J}(\mathbf{r},t)$.
  The distinction between Gaussian diffusion ($\mathcal{D}_G$) and compound-Poisson jumps ($\mathcal{J}_P$) arises from the statistical character of $\mathbf{J}$: dense sources yield Gaussian $S_{JJ}$, while sparse fly-by events produce a L\'evy measure $\nu(\Delta p)$ over momentum kicks.}
  \label{fig:em_mediation}
\end{figure}

The theoretical development shows how integrating out the EM environment within the Keldysh formalism naturally yields a L\'evy--Khintchine form of the ion's master equation~\cite{Sornette2006}: Gaussian diffusion plus compound-Poisson jumps.
The experimental consequences are developed and we outline discriminants that separate Gaussian from non-Gaussian heating, introduce explicit scattering models, and provide inference protocols.
We also identify when the point-impulse approximation breaks down: for fast scatterers with trajectory extent comparable to the trap dimension, the spatial structure of $\mathbf{G}(\mathbf{r})$ induces directional and anisotropic heating.
