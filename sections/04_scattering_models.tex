\section{Observable Signatures and Scattering Models}
\label{sec:scattering-models}
\label{sec:observable_signatures}
The unified Lévy–Khintchine generator (Eq.~\ref{eq:levy-box}) predicts that ion heating arises from a combination of Gaussian diffusion and compound-Poisson jumps.
We develop experimentally distinguishable signatures that allow independent extraction of these components.

\noindent\textbf{(a) Langevin induced-dipole scattering}\quad
\emph{Regime:} Low-energy ion--neutral collisions with polarizable neutral ($E \lesssim 1$~eV).\\
\emph{Cross-section:}
\[
\frac{d\sigma}{d\Omega} =
\frac{\alpha e^2}{8\pi\epsilon_0^2 \mu^2 v^4}\,
\frac{\sin^2\theta}{(1-\cos\theta)^4},
\]
with $\alpha$ the neutral polarizability.
The total Langevin cross-section is $\sigma_{\rm L} = \pi e\sqrt{\alpha/(\epsilon_0 E)} \propto E^{-1/2}$.\\
\emph{Key feature:} Soft divergence at small $\Delta p$ (forward scattering), cut off by quantum diffraction $b_{\min}\sim \hbar/\mu v$.\\
\emph{Consequence:} $\nu(\Delta p)$ has enhanced small-$\Delta p$ weight; CLT $\to$ near-Gaussian heating but with fat tails.\\
\emph{Relevant systems:} Ca$^+$, Sr$^+$, Ba$^+$ with H$_2$, N$_2$, Ar backgrounds.

\textit{Universal behavior:}
The Langevin form is \emph{universal} in that it depends only on the neutral's polarizability $\alpha$, not on molecular details.
This implies that $\nu(\Delta p)$ is parametrized by just $\alpha$ and the velocity distribution $f(v)$, reducing the inference problem to fitting $\lambda$ (or equivalently $n_g$) and $T$ from experimental data.
Departures from universality—resonant charge exchange, short-range chemistry, or high-energy collisions—introduce additional structure in $\nu(\Delta p)$ and require species-specific cross-sections.

\medskip
\noindent\textbf{(b) Hard-sphere scattering}\quad
\emph{Regime:} Classical collisions with effective radius $R$.\\
\emph{Cross-section:}
\[
\frac{d\sigma}{d\Omega} = \frac{R^2}{4\pi}.
\]
\emph{Key feature:} Isotropic scattering, momentum transfer peaked at $|\Delta p|=2\mu v$.\\
\emph{Consequence:} $\nu(\Delta p)$ becomes sharply peaked, yielding bimodal $P(\Delta n)$.\\
\emph{Relevant systems:} Test gases (He, Ne) in controlled background studies.

\medskip
\noindent\textbf{(c) Resonant charge exchange}\quad
\emph{Regime:} Ion--atom collisions with same species (e.g., Ca$^+$ + Ca).\\
\emph{Cross-section:}
\[
\sigma_{\rm CX}(E) \approx \sigma_0\left(1 + \frac{E}{E_0}\right)^{-1/2},
\]
with $\sigma_0$ set by resonant channel strength.
Charge exchange swaps internal states, producing state-dependent heating.

\medskip
\noindent\textbf{(d) Charged dust grains}\quad
\emph{Regime:} Occasional Coulomb interactions with micron-scale charged particles.\\
\emph{Characteristic:} Large impulse events with $|\Delta p| \gg \mu v$, leading to heavy-tailed $\nu(\Delta p)$.
\emph{Observable:} Rare, multi-phonon jumps and long-lived charge-induced fields.

These models feed directly into the inference protocol of Sec.~\ref{sec:inference_protocol}
and the discriminants summarized in Sec.~\ref{sec:discriminants_summary}.
