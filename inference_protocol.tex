\section{Inference Protocol}

\paragraph{Recommended stroboscopic sequence.}
\begin{enumerate}\itemsep0.2em
  \item Prepare motional ground state $\lvert n=0\rangle$ (or a calibrated thermal $\langle n\rangle$).
  \item Wait time $\tau$ (free evolution under noise).
  \item Read out $n$ via sideband thermometry or time-of-flight.
  \item Repeat $N$ times $\Rightarrow$ histogram $P(n\,|\,\tau)$ and time series $n(t)$.
\end{enumerate}
 
\begin{table}[h]
\centering
\renewcommand{\arraystretch}{1.15}
\begin{tabularx}{0.9\textwidth}{>{\raggedright\arraybackslash}X>{\raggedleft\arraybackslash}p{4.2cm}}
\hline
\textbf{Target quantity} & \textbf{Required shots/events (order-of-magnitude)} \\
\hline
Detect $\kappa_3 \neq 0$ at $3\sigma$ & $\sim 10^4$ stroboscopic shots \\
Reconstruct $\nu(\Delta n)$ (5 bins)  & $\sim 10^5$ shots \\
Test heavy-tail exponent $\alpha$      & $\sim 10^6$ events \\
Allan variance rollover near $1/\lambda$ & hours of continuous trace \\
Sideband asymmetry vs.\ $\omega$ map   & $\sim 10^2$ frequency scans \\
\hline
\end{tabularx}
\end{table}
\subsection{Stroboscopic measurement: detailed protocol}

\paragraph{Definition.}
A \emph{stroboscopic measurement} consists of repeated cycles:
\[
\text{Prepare} \to \text{Wait } \tau \to \text{Readout} \to \text{Reset}
\]
with fixed waiting time $\tau$ and $N$ independent repetitions. 
The observable is the phonon number $n$ after free evolution under noise.

\paragraph{Pulse sequence and timing.}

\begin{enumerate}
\item \textbf{Doppler cooling} ($t_{\rm cool} \sim 1{-}2\,\text{ms}$): 
  Bring ion to near ground state via red-detuned cooling laser. 
  Achieved mean phonon number $\bar{n}_{\rm init} \sim 5{-}10$.

\item \textbf{Sideband cooling} (optional, $t_{\rm SBC} \sim 1{-}5\,\text{ms}$): 
  For ground-state preparation $|n=0\rangle$ with fidelity $>95\%$, 
  apply resolved-sideband cooling. 
  Required if discriminating small jumps ($\Delta n \sim 1$).

\item \textbf{Wait interval} ($\tau$, variable): 
  All lasers off; ion evolves under ambient noise. 
  This is the \emph{interrogation time} during which heating accumulates.

\item \textbf{Readout} ($t_{\rm read} \sim 100{-}500\,\mu\text{s}$): 
  Measure $n$ via:
  \begin{itemize}
    \item \emph{Sideband thermometry}: Rabi flop amplitude on red/blue sidebands 
      $\Rightarrow$ extract $\bar{n}$ (ensemble average required, $\sim 20{-}50$ shots per point).
    \item \emph{Time-of-flight imaging}: Release ion from trap, measure kinetic energy 
      spread $\Rightarrow$ single-shot $n$ with resolution $\sim 1{-}2$ phonons.
  \end{itemize}
  
\item \textbf{Reset/dead time} ($t_{\rm dead} \sim t_{\rm cool}$): 
  Return to step 1. Include settling time for trap voltages, laser locks.
\end{enumerate}

\paragraph{Duty cycle and measurement rate.}
The cycle time is
\[
T_{\rm cycle} = t_{\rm cool} + \tau + t_{\rm read} + t_{\rm dead}.
\]
For $N$ shots, total measurement time is $T_{\rm total} = N \cdot T_{\rm cycle}$.
\begin{itemize}
\item \textbf{Short-$\tau$ regime} ($\tau \ll t_{\rm cool}$): 
  Duty cycle $\sim \tau/t_{\rm cool} \ll 1$; 
  most time spent cooling. 
  Use when probing fast dynamics (large $\lambda$) or building $P(n|\tau)$ at multiple $\tau$.

\item \textbf{Long-$\tau$ regime} ($\tau \gg t_{\rm cool}$): 
  Duty cycle $\sim \tau/(t_{\rm cool}+\tau)$; 
  efficient statistics. 
  Use when event rate is low ($\lambda \ll 1/\tau$) or testing diffusive limit.
\end{itemize}
Typical experimental constraints: $t_{\rm cool} \sim 5\,\text{ms}$, 
$\tau \in [10\,\mu\text{s}, 10\,\text{s}]$ depending on regime.

\paragraph{Optimal $\tau$ selection strategy.}
The choice of $\tau$ depends on the target regime (Sec.~2.6):
\begin{itemize}
\item \textbf{Diffusive limit} ($\lambda \tau \gg 1$): 
  Choose $\tau$ such that $\langle n(\tau) \rangle \sim 5{-}20$ 
  (well above readout noise, well below trap anharmonicity). 
  Scan $\tau$ to verify $\langle n \rangle \propto \tau$ (signature of constant heating rate).

\item \textbf{Intermediate regime} ($\lambda \tau \sim 1{-}10$): 
  Choose $\tau \sim 1/\lambda$ where Allan variance shows rollover. 
  Requires prior estimate of $\lambda$ from heating-rate measurements 
  or iterative refinement.

\item \textbf{Poisson limit} ($\lambda \tau \ll 1$): 
  Choose $\tau$ short enough that $P(\text{no jumps}) \sim e^{-\lambda\tau} \sim 0.5{-}0.9$. 
  Scan $\tau$ logarithmically to map out waiting-time distribution.
\end{itemize}
In practice, measure at $\sim 5{-}10$ values of $\tau$ spanning $[\tau_{\rm min}, \tau_{\rm max}]$ 
with $\tau_{\rm max}/\tau_{\rm min} \sim 10{-}100$ to cover crossover.

\paragraph{Single-shot vs.\ ensemble measurements.}
\begin{itemize}
\item \textbf{Single-shot readout} (time-of-flight, electron shelving + imaging): 
  Directly yields $\{n_i\}_{i=1}^N$ for each shot. 
  Enables change-point inference, jump-size histograms, waiting-time analysis. 
  Required resolution: $\sim 1{-}2$ phonons.

\item \textbf{Ensemble averaging} (sideband thermometry): 
  Yields $\langle n(\tau) \rangle$ and variance $\sigma_n^2(\tau)$ 
  but not individual trajectories. 
  Can still extract $\kappa_2, \kappa_3, \kappa_4$ via repeated blocks; 
  cannot resolve individual jumps.
\end{itemize}
For full discriminatory power (Table~1), single-shot readout is preferred.

\paragraph{Systematic corrections.}
\begin{itemize}
\item \textbf{Imperfect ground-state preparation}: 
  If $\bar{n}_{\rm init} \neq 0$, measure $n$ immediately after cooling (zero wait) 
  to characterize initial distribution $P_{\rm init}(n)$. 
  Subtract or deconvolve from $P(n|\tau)$.

\item \textbf{Measurement-induced heating}: 
  Readout light can induce additional heating ($\sim 0.1{-}1$ phonon/shot). 
  Calibrate by varying readout duration; extrapolate to zero exposure.

\item \textbf{Finite detection efficiency}: 
  Photon collection/shelving readout has $\eta < 1$. 
  Model as binomial sampling; correct via likelihood rescaling 
  or use thresholded estimators robust to $\eta$.
\end{itemize}

\paragraph{Example experimental parameters (Ca$^+$, $\omega_t/2\pi = 1\,\text{MHz}$).}
\begin{itemize}
\item Intermediate regime ($\lambda \sim 0.1{-}1\,\text{s}^{-1}$): 
  $\tau \in [1, 10]\,\text{s}$, $N = 10^4$ shots $\Rightarrow$ 
  $T_{\rm total} \sim 30{-}300\,\text{hours}$ (parallelizable with ion arrays).
\item Diffusive regime (Langevin at $p \sim 10^{-10}\,\text{mbar}$): 
  $\tau \in [0.1, 10]\,\text{ms}$, $N = 10^5$ shots $\Rightarrow$ 
  $T_{\rm total} \sim 10{-}100\,\text{minutes}$.
\end{itemize}

\subsection{Multi-$\tau$ scan for regime identification}

To distinguish diffusive, intermediate, and Poisson limits \emph{without prior knowledge}, 
perform a logarithmic $\tau$ scan:

\begin{enumerate}
\item Choose $\tau_j = \tau_0 \cdot 2^j$ for $j = 0, 1, \ldots, J$ 
  with $\tau_0 \sim 10\,\mu\text{s}$ and $J \sim 10{-}15$ 
  (spanning $\tau_0$ to $\tau_0 \cdot 2^{15} \sim 300\,\text{ms}$).

\item At each $\tau_j$, collect $N_j = N_{\rm target}$ stroboscopic shots. 
  Allocate measurement time equally: $N_j \cdot T_{\rm cycle}(\tau_j) \approx \text{const}$.

\item Compute for each $\tau_j$:
  \begin{itemize}
    \item Mean heating: $\langle n(\tau_j) \rangle$
    \item Variance: $\sigma_n^2(\tau_j)$
    \item Skewness: $\kappa_3(\tau_j) / \kappa_2(\tau_j)^{3/2}$
  \end{itemize}

\item \textbf{Regime identification from scaling}:
  \begin{itemize}
    \item \emph{Diffusive}: $\langle n \rangle \propto \tau$, 
      $\sigma_n^2 \propto \tau$, $\kappa_3 \to 0$.
    \item \emph{Intermediate}: $\langle n \rangle \propto \tau$ 
      but $\kappa_3/\kappa_2^{3/2}$ non-zero and $\tau$-dependent; 
      $\sigma_n^2/\langle n \rangle$ shows rollover.
    \item \emph{Poisson}: $\langle n \rangle \propto \lambda\tau$ 
      with measurable jumps; $P(n|\tau)$ non-Gaussian.
  \end{itemize}

\item Use the identified regime to refine $\tau$ selection for high-statistics runs.
\end{enumerate}

\paragraph{Systematic error sources (and mitigations).}
\begin{itemize}\itemsep0.2em
  \item \textbf{Readout noise} smears $P(\Delta n)$: calibrate with known coherent kicks and deconvolve.
  \item \textbf{Motional dephasing} during $\tau$: ensure $\tau \ll 1/\Gamma_{\rm dephase}$ or include dephasing in the likelihood model.
  \item \textbf{Multi-ion coupling}: analyze in normal-mode basis; report mode-resolved $\nu(\Delta n)$.
  \item \textbf{Background drift / non-stationarity}: use sliding-window estimators for $\lambda(t)$, $\langle \Delta n \rangle(t)$.
\end{itemize}

\paragraph{Estimation outputs (minimum set).}
\begin{itemize}\itemsep0.2em
  \item Cumulants $\kappa_2(\tau)$, $\kappa_3(\tau)$, $\kappa_4(\tau)$; Gaussian vs.\ non-Gaussian decision.
  \item Change-point inference of jump times $\{t_k\}$ and sizes $\{\Delta n_k\}$; empirical $\hat{\nu}(\Delta n)$.
  \item Rollover in $\sigma_A^2(\tau)$ and sideband asymmetry vs.\ $\omega$ to constrain $|\mathbf{G}(\omega)|^2$.
\end{itemize}% Inference protocol placeholder